\chapter{Conclusion and Future Enhancements}
This project is developed to minimize the chaos ensued and time consumed in managing a lot of books manually by an end user. This is done by embedding all the necessary data in a concise database and creating an apt interface. A future version of this project would incorporate multiple end users thus making this a social media. Suppose you have a spare hour or so in an airport and you want to do some light reading, this time is best spent reading and not choosing and searching for the book; here comes the book cataloguing system, this can be used to select an ebook after reading through the reviews. \\[0.2in]
The Book Cataloguing System is a rather personal project. I am an avid reader and wanted a tool to minimize my selection time and also a tool to store all the details of my personal book copies. I am thankful for being
provided this great opportunity to work on it.I am already implementing this to manage my books. As already mentioned, this project has gone through extensive research work. On the basis of the research work, we have successfully
designed and implemented The Book Cataloguing System. The world is becoming digital. Almost everything in the real world has it's counterpart in the virtual world. The Book Cataloguing system brings your personal book collection into the digital context.  \\[0.2in]

The most valuable future looks are following below:
\begin{itemize}

\item Having a globally accessible database of books so you can access it from anywhere.
\item A child safety feature that doesn't allow people under certain ages to access certain books
\item A social media pivot that makes this a place where people can review and rate books and add them to their shelves. A global marketplace for books.

\end{itemize}
