\chapter{Implementation}
\section{FORMS}
\subsection{Home Page}
\begin{lstlisting}
using System;
using System.Collections.Generic;
using System.ComponentModel;
using System.Data;
using System.Drawing;
using System.Linq;
using System.Text;
using System.Threading.Tasks;
using System.Windows.Forms;


namespace BookCataloguing
{
    public partial class Form1 : Form
    {
        Form2 f2 = new Form2();
        Form3 f3 = new Form3();
        Form4 f4 = new Form4();
        Form5 f5 = new Form5();
        Form6 f6 = new Form6();
        Form7 f7 = new Form7();
        Form8 f8 = new Form8();
        Form9 f9 = new Form9();

        public Form1()
        {
            InitializeComponent();
        }

        private void button1_Click(object sender, EventArgs e)
        {
            try
            {
                f2.ShowDialog();
            }
            catch(Exception exception)
            {
                new Form2().ShowDialog();
            }
           

        }

        

        private void button2_Click_1(object sender, EventArgs e)
        {
            try
            {
                f3.ShowDialog();
            }
            catch (Exception exceptio)
            {
                new Form3().ShowDialog();
            }
        }

        private void button3_Click_1(object sender, EventArgs e)
        {
            try
            {
                f4.ShowDialog();
            }
            catch (Exception exceptin)
            {
                new Form4().ShowDialog();
            }
        }

        private void button4_Click_1(object sender, EventArgs e)
        {
            try
            {
                f5.ShowDialog();
            }
            catch (Exception excepton)
            {
                new Form5().ShowDialog();
            }
        }


        private void button6_Click(object sender, EventArgs e)
        {
            try
            {
                f6.ShowDialog();
            }
            catch (Exception exception)
            {
                new Form6().ShowDialog();
            }
        }

        private void button8_Click(object sender, EventArgs e)
        {
            this.Close();
        }

        private void button5_Click(object sender, EventArgs e)
        {
           
        }

        private void button7_Click(object sender, EventArgs e)
        {
           
        }

        private void button10_Click(object sender, EventArgs e)
        {
            try
            {
                f8.ShowDialog();
            }
            catch (Exception xceptio)
            {
                new Form8().ShowDialog();
            }
        }

        private void button9_Click(object sender, EventArgs e)
        {
            try
            {
                f9.ShowDialog();
            }
            catch (Exception xceptio)
            {
                new Form9().ShowDialog();
            }
        }

        private void button5_Click_1(object sender, EventArgs e)
        {
            try
            {
                f7.ShowDialog();
            }
            catch (Exception exption)
            {
                new Form7().ShowDialog();
            }
        }

        private void button7_Click_1(object sender, EventArgs e)
        {
            this.Close();
        }
    }
}
\end{lstlisting}
\subsection{Book information}
\begin{lstlisting}
using System;
using System.Collections.Generic;
using System.ComponentModel;
using System.Data;
using System.Drawing;
using System.Linq;
using System.Text;
using System.Threading.Tasks;
using System.Windows.Forms;
using System.Data.SqlClient;
using System.Diagnostics;
namespace BookCataloguing
{
    public partial class Form2 : Form
    {

        SqlConnection con = new SqlConnection("Data Source=(LocalDB)\\MSSQLLocalDB;AttachDbFilename=|DataDirectory|\\Database1.mdf;Integrated Security=True");
        SqlCommand cmd;
        SqlDataReader Dr1;

        

        public void getauthrat()


        {
           
            
            con.Open();
             String b = label8.Text;

            string syntax = "SELECT a.authname, r.rating FROM authors a, rating r WHERE a.bid="+b+" and r.bid="+b;
            cmd = new SqlCommand(syntax, con);
            
            
                Dr1 = cmd.ExecuteReader();
                Dr1.Read();
                
            
           // catch(Exception e)
           // {
              //  MessageBox.Show("sql injection error");
            //}
            label3.Text = Dr1[0].ToString();
            label4.Text = Dr1[1].ToString();
            con.Close();
            // syntax = "";
        }

        public Form2()
        {
            InitializeComponent();
        }

       public void img()
        {
            string a = label7.Text;
            pictureBox1.ImageLocation = a;
            
        }
       
        
        private void button1_Click_1(object sender, EventArgs e)
        {
            this.Close();
        }

        private void Form2_Load(object sender, EventArgs e)
        {
            // TODO: This line of code loads data into the 'database1DataSet.location' table. You can move, or remove it, as needed.
            this.locationTableAdapter.Fill(this.database1DataSet.location);

            // TODO: This line of code loads data into the 'database1DataSet.book' table. You can move, or remove it, as needed.
            this.bookTableAdapter.Fill(this.database1DataSet.book);
           
           

        }

        private void button5_Click(object sender, EventArgs e)
        {
            con.Open();

            String b = label8.Text;
            string syntax = "SELECT buylink FROM link WHERE bid="+b;
            cmd = new SqlCommand(syntax, con);
            Dr1 = cmd.ExecuteReader();
            Dr1.Read();
            Process.Start(Dr1[0].ToString());
            
            con.Close();

        }

        

        private void listBox1_MouseMove(object sender, MouseEventArgs e)
        {
            img();
            getauthrat();
            Dr1.Equals("");
        }

        private void button3_Click(object sender, EventArgs e)
        {
            SqlDataReader Dr1;
            con.Open();

            String bidlink = label8.Text;
            string syntax = "SELECT location FROM location WHERE bid=" + bidlink;
            cmd = new SqlCommand(syntax, con);
            Dr1 = cmd.ExecuteReader();
            Dr1.Read();
            string loc = Dr1[0].ToString();
            loc.Trim();
            Process.Start(loc);
           
            con.Close();
            return;
        }

        private void label9_Click(object sender, EventArgs e)
        {

        }
    }
}
\end{lstlisting}
\subsection{Author page}
\begin{lstlisting}
using System;
using System.Collections.Generic;
using System.ComponentModel;
using System.Data;
using System.Drawing;
using System.Linq;
using System.Text;
using System.Threading.Tasks;
using System.Windows.Forms;
using System.Data.SqlClient;
namespace BookCataloguing
{
    public partial class Form3 : Form
    {
        SqlConnection con = new SqlConnection("Data Source=(LocalDB)\\MSSQLLocalDB;AttachDbFilename=|DataDirectory|\\Database1.mdf;Integrated Security=True");
        SqlCommand cmd;
        SqlDataReader Dr1;

        public Form3()
        {
            InitializeComponent();
        }
        public void bkld(object sender, EventArgs e)
        {
            try
            {
                cmd = new SqlCommand("auth1_SP", con);
                cmd.CommandType = CommandType.StoredProcedure;

                cmd.Parameters.AddWithValue("@authname", label1.Text);
                SqlDataAdapter DA = new SqlDataAdapter(cmd);
                DataSet DS = new DataSet();
                DA.Fill(DS);

                con.Open();
                try
                {
                    cmd.ExecuteNonQuery();


                }
                catch (Exception ex)
                {
                    MessageBox.Show("<<<INVALID SQL OPERATION>>> \n" + ex);

                }
                con.Close();

                dataGridView1.DataSource = DS.Tables[0];
                this.dataGridView1.Columns[0].AutoSizeMode = DataGridViewAutoSizeColumnMode.DisplayedCells;
                this.dataGridView1.Columns[1].AutoSizeMode = DataGridViewAutoSizeColumnMode.Fill;
               




            }
            catch (Exception ex)
            {
                MessageBox.Show(" " + ex);
            }
            

          //  String b = label3.Text;
           // string syntax = "SELECT authimgurl FROM authors WHERE bid=" + b; 
           // cmd = new SqlCommand(syntax, con);


           // Dr1 = cmd.ExecuteReader();
          //  Dr1.Read();


           
           // label2.Text = Dr1[0].ToString();
           
            
            pictureBox1.ImageLocation = label2.Text;

        }




        private void button2_Click_1(object sender, EventArgs e)
        {
            this.Close();
        }

        private void button1_Click(object sender, EventArgs e)
        {
            
        }

        private void Form3_Load(object sender, EventArgs e)
        {
            // TODO: This line of code loads data into the 'database1DataSet1.authors' table. You can move, or remove it, as needed.
            this.authorsTableAdapter1.Fill(this.database1DataSet1.authors);
            // TODO: This line of code loads data into the 'database1DataSet.authors' table. You can move, or remove it, as needed.
            this.authorsTableAdapter.Fill(this.database1DataSet.authors);

        }

        private void listBox1_MouseMove(object sender, MouseEventArgs e)
        {
            string a = label2.Text;
            pictureBox1.ImageLocation = a;
            try
            {
                cmd = new SqlCommand("auth1_SP", con);
                cmd.CommandType = CommandType.StoredProcedure;

                cmd.Parameters.AddWithValue("@authname", label1.Text);
                SqlDataAdapter DA = new SqlDataAdapter(cmd);
                DataSet DS = new DataSet();
                DA.Fill(DS);

                con.Open();
                try
                {
                    cmd.ExecuteNonQuery();


                }
                catch (Exception ex)
                {
                    MessageBox.Show("<<<INVALID SQL OPERATION>>> \n" + ex);

                }
                con.Close();

                dataGridView1.DataSource = DS.Tables[0];
                this.dataGridView1.Columns[0].AutoSizeMode = DataGridViewAutoSizeColumnMode.DisplayedCells;
                this.dataGridView1.Columns[1].AutoSizeMode = DataGridViewAutoSizeColumnMode.Fill;

               



            }
            catch (Exception ex)
            {
                MessageBox.Show(" " + ex);
            }

        }
    }
}
\end{lstlisting}
\subsection{Genre page}
\begin{lstlisting}
using System;
using System.Collections.Generic;
using System.ComponentModel;
using System.Data;
using System.Drawing;
using System.Linq;
using System.Text;
using System.Threading.Tasks;
using System.Windows.Forms;
using System.Data.SqlClient;

namespace BookCataloguing
{
    public partial class Form4 : Form
    {
        SqlConnection con = new SqlConnection("Data Source=(LocalDB)\\MSSQLLocalDB;AttachDbFilename=|DataDirectory|\\Database1.mdf;Integrated Security=True");
        SqlCommand cmd;
        public Form4()
        {
            InitializeComponent();
        }

        private void button2_Click_1(object sender, EventArgs e)
        {
            this.Close();
        }

        private void button1_Click(object sender, EventArgs e)

        {
            try
            {
                cmd = new SqlCommand("gen_sp", con);
                cmd.CommandType = CommandType.StoredProcedure;

                cmd.Parameters.AddWithValue("@gname", textBox1.Text);
                SqlDataAdapter DA = new SqlDataAdapter(cmd);
                DataSet DS = new DataSet();
                DA.Fill(DS);

                con.Open();
                try
                {
                    cmd.ExecuteNonQuery();


                }
                catch (Exception ex)
                {
                    MessageBox.Show("<<<INVALID SQL OPERATION>>> \n" + ex);

                }
                con.Close();

                dataGridView1.DataSource = DS.Tables[0];
                this.dataGridView1.Columns[0].AutoSizeMode = DataGridViewAutoSizeColumnMode.DisplayedCells;
                this.dataGridView1.Columns[1].AutoSizeMode = DataGridViewAutoSizeColumnMode.Fill;
                this.dataGridView1.Columns[2].AutoSizeMode = DataGridViewAutoSizeColumnMode.DisplayedCells;




            }
            catch (Exception ex)
            {
                MessageBox.Show(" " + ex);
            }

        }

        private void panel4_Paint(object sender, PaintEventArgs e)
        {

        }

        private void button7_Click(object sender, EventArgs e)
        {

        }

        private void button2_Click(object sender, EventArgs e)
        {
            this.Close();
        }
    }
}


\end{lstlisting}
\subsection{Award page}
\begin{lstlisting}
using System;
using System.Collections.Generic;
using System.ComponentModel;
using System.Data;
using System.Data.SqlClient;
using System.Drawing;
using System.Linq;
using System.Text;
using System.Threading.Tasks;
using System.Windows.Forms;

namespace BookCataloguing
{
    public partial class Form5 : Form
    {
        SqlConnection con = new SqlConnection("Data Source=(LocalDB)\\MSSQLLocalDB;AttachDbFilename=|DataDirectory|\\Database1.mdf;Integrated Security=True");
        SqlCommand cmd;
        public Form5()
        {
            InitializeComponent();
        }

        

        private void button2_Click_1(object sender, EventArgs e)
        {
            this.Close();
        }

        private void Form5_Load(object sender, EventArgs e)
        {
            // TODO: This line of code loads data into the 'database1DataSet.awards' table. You can move, or remove it, as needed.
            this.awardsTableAdapter.Fill(this.database1DataSet.awards);
            

        }

        private void button1_Click(object sender, EventArgs e)
        {
            try
            {
                cmd = new SqlCommand("awards_sp", con);
                cmd.CommandType = CommandType.StoredProcedure;

                cmd.Parameters.AddWithValue("@awards", comboBox1.Text);
                SqlDataAdapter DA = new SqlDataAdapter(cmd);
                DataSet DS = new DataSet();
                DA.Fill(DS);

                con.Open();
                try
                {
                    cmd.ExecuteNonQuery();


                }
                catch (Exception ex)
                {
                    MessageBox.Show("<<<INVALID SQL OPERATION>>> \n" + ex);

                }
                con.Close();

                dataGridView1.DataSource = DS.Tables[0];
                this.dataGridView1.Columns[0].AutoSizeMode = DataGridViewAutoSizeColumnMode.DisplayedCells;
                this.dataGridView1.Columns[1].AutoSizeMode = DataGridViewAutoSizeColumnMode.Fill;
                




            }
           catch (Exception ex)
           {
                MessageBox.Show(" " + "ok");
            }
        }

        private void button2_Click(object sender, EventArgs e)
        {
            this.Close();
        }

        private void comboBox1_SelectedIndexChanged(object sender, EventArgs e)
        {

        }

        private void fillByToolStripButton_Click(object sender, EventArgs e)
        {
            try
            {
                this.awardsTableAdapter.FillBy(this.database1DataSet.awards);
            }
            catch (System.Exception ex)
            {
                System.Windows.Forms.MessageBox.Show(ex.Message);
            }

        }

        private void button2_Click_2(object sender, EventArgs e)
        {
            this.Close();
        }
    }
}
\end{lstlisting}
\subsection{Rating page}
\begin{lstlisting}
using System;
using System.Collections.Generic;
using System.ComponentModel;
using System.Data;
using System.Drawing;
using System.Linq;
using System.Text;
using System.Threading.Tasks;
using System.Windows.Forms;
using System.Data.SqlClient;
namespace BookCataloguing
{
    public partial class Form6 : Form
    {
        SqlConnection con = new SqlConnection("Data Source=(LocalDB)\\MSSQLLocalDB;AttachDbFilename=|DataDirectory|\\Database1.mdf;Integrated Security=True");
        SqlCommand cmd;
        public Form6()
        {
            InitializeComponent();
        }

        private void button2_Click(object sender, EventArgs e)
        {
            this.Close();

        }

        private void panel4_Paint(object sender, PaintEventArgs e)
        {

        }

        private void button5_Click(object sender, EventArgs e)
        {
            try
            {
                SqlCommand cmd = new SqlCommand("rating_sp", con);
                cmd.CommandType = CommandType.StoredProcedure;

                cmd.Parameters.AddWithValue("@rating", textBox1.Text);
                SqlDataAdapter DA = new SqlDataAdapter(cmd);
                DataSet DS = new DataSet();
                DA.Fill(DS);

                con.Open();
                try
                {
                    cmd.ExecuteNonQuery();


                }
                catch (Exception ex)
                {
                    MessageBox.Show("<<<INVALID SQL OPERATION>>> \n" + ex);

                }
                con.Close();

                dataGridView1.DataSource = DS.Tables[0];
                this.dataGridView1.Columns[0].AutoSizeMode = DataGridViewAutoSizeColumnMode.DisplayedCells;
                this.dataGridView1.Columns[1].AutoSizeMode = DataGridViewAutoSizeColumnMode.Fill;
                this.dataGridView1.Columns[2].AutoSizeMode = DataGridViewAutoSizeColumnMode.DisplayedCells;




            }
            catch (Exception ex)
            {
                MessageBox.Show(" " + ex);
            }

        }

        private void button2_Click_1(object sender, EventArgs e)
        {
            this.Close();
        }
    }
}
\end{lstlisting}
\subsection{Log page}
\begin{lstlisting}
using System;
using System.Collections.Generic;
using System.ComponentModel;
using System.Data;
using System.Drawing;
using System.Linq;
using System.Text;
using System.Threading.Tasks;
using System.Windows.Forms;

namespace BookCataloguing
{
    public partial class Form7 : Form
    {
        public Form7()
        {
            InitializeComponent();
        }

        private void Form7_Load(object sender, EventArgs e)
        {
            // TODO: This line of code loads data into the 'database1DataSet1.logdetails' table. You can move, or remove it, as needed.
            this.logdetailsTableAdapter.Fill(this.database1DataSet1.logdetails);

        }

        private void button2_Click(object sender, EventArgs e)
        {
            this.Close();
        }

      
    }
}
\end{lstlisting}
\subsection{Insert page}
\begin{lstlisting}
using System;
using System.Collections.Generic;
using System.ComponentModel;
using System.Data;
using System.Data.SqlClient;
using System.Drawing;
using System.Linq;
using System.Text;
using System.Threading.Tasks;
using System.Windows.Forms;

namespace BookCataloguing
{
    public partial class Form8 : Form
    {
        
        public Form8()
        {
            InitializeComponent();
        }
        
        private void label4_Click(object sender, EventArgs e)
        {

        }

        private void button1_Click(object sender, EventArgs e)
        {
            SqlConnection con = new SqlConnection("Data Source=(LocalDB)\\MSSQLLocalDB;AttachDbFilename=|DataDirectory|\\Database1.mdf;Integrated Security=True");
            SqlCommand cmd = new SqlCommand("insert_SP", con);
            cmd.CommandType = CommandType.StoredProcedure;
            cmd.Parameters.AddWithValue("@bid", textBox14.Text);
            cmd.Parameters.AddWithValue("@bname", textBox1.Text);
            cmd.Parameters.AddWithValue("@bimgurl", textBox4.Text);
            cmd.Parameters.AddWithValue("@bpub", textBox9.Text);
            cmd.Parameters.AddWithValue("@bpubyr", textBox10.Text);
            cmd.Parameters.AddWithValue("@breview", textBox8.Text);
            cmd.Parameters.AddWithValue("@rating", textBox2.Text);

            cmd.Parameters.AddWithValue("@authname", textBox6.Text);
            cmd.Parameters.AddWithValue("@awards", textBox5.Text);
            cmd.Parameters.AddWithValue("@gname", textBox3.Text);
            cmd.Parameters.AddWithValue("@agepref", textBox13.Text);
            cmd.Parameters.AddWithValue("@buylink", textBox12.Text);
            cmd.Parameters.AddWithValue("@location", textBox11.Text);
            cmd.Parameters.AddWithValue("@authimgurl", textBox7.Text);
            //  cmd.Parameters.AddWithValue("@", maskedTextBox13.Text);
            //  cmd.Parameters.AddWithValue("@rating", maskedTextBox13.Text);
            float rat=float.Parse(textBox2.Text);
            if (rat>5)
            {
                MessageBox.Show("ENTER RATING LESS THAN 5");
            }

            con.Open();
            try
            {
                if (rat > 5)
                {
                   // Form8 f8 = new Form8();
                    MessageBox.Show("ENTER RATING LESS THAN 5");
                    
                    new Form8().ShowDialog();
                    this.Close();
                    return;
                }
                cmd.ExecuteNonQuery();
                this.Close();
            }
            catch (Exception ex)
            {
                MessageBox.Show("    <<<<<INVALID SQL OPERATION>>>\n" + ex);
            }
            con.Close();
        }

        private void textBox1_TextChanged(object sender, EventArgs e)
        {

        }

        private void label11_Click(object sender, EventArgs e)
        {

        }

        private void label10_Click(object sender, EventArgs e)
        {

        }

        private void button2_Click(object sender, EventArgs e)
        {
            this.Close();
        }
    }
}
\end{lstlisting}
\subsection{Delete page}
\begin{lstlisting}
using System;
using System.Collections.Generic;
using System.ComponentModel;
using System.Data;
using System.Drawing;
using System.Linq;
using System.Text;
using System.Threading.Tasks;
using System.Windows.Forms;
using System.Data.SqlClient;

namespace BookCataloguing
{
    public partial class Form9 : Form
    {

        SqlConnection con = new SqlConnection("Data Source=(LocalDB)\\MSSQLLocalDB;AttachDbFilename=|DataDirectory|\\Database1.mdf;Integrated Security=True");
       
        SqlDataReader Dr1;

        public Form9()
        {
            InitializeComponent();
        }

        private void button1_Click(object sender, EventArgs e)
        {

        }

        private void Form9_Load(object sender, EventArgs e)
        {
            // TODO: This line of code loads data into the 'database1DataSet.book' table. You can move, or remove it, as needed.
            this.bookTableAdapter1.Fill(this.database1DataSet.book);
            // TODO: This line of code loads data into the 'database1DataSet1.book' table. You can move, or remove it, as needed.
            this.bookTableAdapter.Fill(this.database1DataSet1.book);
           
        }

        private void textBox1_TextChanged(object sender, EventArgs e)
        {

        }

        private void button3_Click(object sender, EventArgs e)
        {
            
            try
            {
               
               SqlCommand cmd = new SqlCommand("delete_sp", con);
                cmd.CommandType = CommandType.StoredProcedure;

                cmd.Parameters.AddWithValue("@bid", label3.Text);

                con.Open();
               
                try
                {
                    cmd.ExecuteNonQuery();
                }
                catch (Exception ex)
                {
                    MessageBox.Show("  <<<<<<<INVALID SQL OPERATION\n" + ex);
                }
                con.Close();
\end{lstlisting}

\section{STORED PROCEDURES}
\subsection{Book information related to a particular author}
\begin{lstlisting}
CREATE PROCEDURE [dbo].auth1_SP
	@authname nchar(50)
	
AS
	SELECT b.bname,a.authname from
	book b,authors a
	where a.authname=@authname and
	b.bid=a.bid
RETURN 0

\end{lstlisting}
\subsection{Book information related to a particular award}
\begin{lstlisting}
CREATE PROCEDURE [dbo].awards_sp
	@awards nchar(25)
AS
	SELECT b.bid,b.bname from 
	book b, awards a  where
	b.bid=a.bid and a.awards=@awards
RETURN 0
\end{lstlisting}
\subsection{Book information related to a particular genre}
\begin{lstlisting}
CREATE PROCEDURE [dbo].gen_sp
	@gname nchar(20)
AS
	SELECT b.bname,a.authname,g.gname
	from book b,authors a, genre g
	where b.bid=a.bid and b.bid=g.bid
	and g.gname=@gname
RETURN 0
\end{lstlisting}
\subsection{Book information based on rating}
\begin{lstlisting}
CREATE PROCEDURE [dbo].rating_sp
	@rating float
	
AS
	SELECT b.bname, r.rating, g.agepref FROM 
	book b, rating r, genre g WHERE b.bid=r.bid and b.bid=g.bid
	and r.rating>=@rating
RETURN 0
\end{lstlisting}
\subsection{Insert new book}
\begin{lstlisting}
CREATE PROCEDURE [dbo].insert_SP
	@bid int = 0,
	@authname nchar(50),
	@authimgurl varchar(200),
	@awards nchar(25),
	@bimgurl varchar(200),
	@bpub nchar(50),
	@bpubyr nchar(4),
	@breview varchar(MAX),
	@gname nchar(20),
	@agepref nchar(10),
	@buylink varchar(200),
	@location varchar(200),
	@rating float,


	@bname nchar(50)
	
AS
	insert into book(bid,bname,bimgurl,bpub,bpubyr,breview)values(@bid,@bname,@bimgurl,@bpub,@bpubyr,@breview)
	insert into rating(bid,rating)values(@bid,@rating)
	insert into authors(bid,authname,authimgurl)values(@bid,@authname,@authimgurl)
	insert into awards(bid,awards)values(@bid,@awards)
	insert into link(bid,buylink)values(@bid,@buylink)
	insert into location (bid,location)values(@bid,@location)
	insert into genre(bid,gname,agepref)values(@bid,@gname,@agepref)

RETURN 0
\end{lstlisting}
\subsection{Delete book}
\begin{lstlisting}
CREATE PROCEDURE [dbo].delete_sp
	@bid int 
	
AS
	DELETE FROM authors WHERE bid = @bid
	DELETE FROM awards WHERE bid = @bid
	DELETE FROM link WHERE bid = @bid
	DELETE FROM rating WHERE bid = @bid
	DELETE FROM genre WHERE bid = @bid
	DELETE FROM location WHERE bid = @bid
	DELETE FROM book WHERE bid = @bid
 RETURN 0
\end{lstlisting}
\section{TRIGGER}
\subsection{Log information of a book}
\begin{lstlisting}
CREATE TRIGGER tr_book_forinsert
	ON book
	FOR INSERT
	AS
	BEGIN
		SET NOCOUNT ON
		declare @booknm nchar(50)
		

		select @booknm = bname from inserted
		

		insert into logdetails values('Book with id'+cast(@booknm as nchar(50))+'inserted on'+ cast(GETDATE() as varchar(25)));
	END
\end{lstlisting}





